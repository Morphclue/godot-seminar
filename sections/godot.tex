\section{Godot Engine}
On February 10, 2014, the Godot Engine went open source on GitHub under the MIT license.
The Godot Engine is relatively new compared to popular game engines like Unity and Unreal Engine, which were released in 2005 and 1998, respectively\textcolor{red}{[?]}.
It was released as open source under the MIT license on GitHub on February 10, 2014\footnote{The first commit containing the source code can be found under the SHA:\\ 0b806ee0fc9097fa7bda7ac0109191c9c5e0a1ac}\cite{godot-repository}.
The first stable release was published in December 2014\cite{godot-release}.
Considering the different release dates, it should be clear that the Godot engine has had less time to build a large ecosystem compared to the other two game engines mentioned.\\

The Godot Engine is cross-platform and supports the development of 2D or 3D games\cite{godot-engine}.
Video games can be exported for PC, mobile devices or web platforms.
These are written with a programming language specially developed for the Godot Engine called GDScript.
GDScript is a dynamically typed programming language which is syntactically similar to Python.
However, the Godot engine also officially supports C\texttt{\#} and includes GDNative, a technology for creating bindings to other languages\cite{godot-gdnative}.
These officially include C\texttt{++} and C.
Unofficially support for languages such as D, Kotlin, Nim, Python and Rust have been developed by the community.

\section{Godot Engine}
On February 10, 2014, the Godot Engine was released on GitHub.
The Godot Engine is relatively new compared to popular game engines like Unity and Unreal Engine, which were released in 2005 and 1998, respectively\cite{unity-release}\cite{unreal-release}.
It was released as open source under the MIT license on GitHub on February 10, 2014\footnote{The first commit containing the source code can be found under the SHA:\\ 0b806ee0fc9097fa7bda7ac0109191c9c5e0a1ac}\cite{godot-repository}.
The first stable release was published in December 2014\cite{godot-release}.
Considering the different release dates, it should be clear that the Godot Engine has had less time to build a large ecosystem compared to the other two game engines mentioned.\\

The Godot Engine is cross-platform and supports the development of 2D or 3D games\cite{godot-engine}.
Video games can be exported for PC, mobile devices or web platforms.
These are written with a programming language specially developed for the Godot Engine called GDScript.
GDScript is a dynamically typed programming language which is syntactically similar to Python.
However, the Godot Engine also officially supports C\texttt{\#} and includes GDNative, a technology for creating bindings to other languages\cite{godot-gdnative}.
These officially include C\texttt{++} and C.
Unofficially support for languages such as D, Kotlin, Nim, Python and Rust have been developed by the community.\\

It is important to mention that the Godot Engine seems to be interesting not only for smaller companies, but also for companies like Microsoft, Epic Games and Meta's Reality Labs.
Microsoft donated \$24,000 to the Godot team in 2017 for implementing C\texttt{\#} as a programming language\cite{godot-csharp}.
In 2020, the Godot Engine received the Epic MegaGrant of \$250,000, which is distributed by Epic Games to projects that expand the open source 3D graphics ecosystem\cite{godot-megagrant}.
This case in particular is interesting, because Epic \linebreak Games is also the company responsible for the development of the Unreal Engine.
A possible reason for funding other game engines could be that the Unreal Engine might benefit from the new technologies.
Especially an MIT license like Godot Engine has could allow taking over intellectual property.
However, this doesn't have to be negative, since both sides and the industry in general could benefit from it.
December 2020 and 2021 Godot Engine received a donation from Meta's Reality Labs\cite{godot-facebook-reality}\cite{godot-meta-reality}.
The money will be used specifically for development in the field of extended reality (XR).
There is no information about the amount of the donation.
This was confirmed in a Reddit comment by project manager Rémi Verschelde (akien-mga)\cite{reddit-companies-akien}.
There, Rémi also described that the team actively approaches companies and not the other way around.

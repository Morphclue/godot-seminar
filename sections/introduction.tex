\section{Introduction}
Video games have not been developed from scratch for a longer period of time.
As early as the 1980s, several companies began using their own in-house engines.
For example, in Super Mario Bros (1985), Nintendo used a fast scrolling game engine that was originally developed for Excitebike (1984)\cite{history-digital-games}.
The idea of licensing game engines to other companies as a separate product has been around since the mid-90's.
At that time, games like Doom (1993) were being developed using the id Tech game engine\textcolor{red}{[?]}.
This engine was specifically designed for 3D shooters, although this iteration was not yet a full 3D game engine.
Over time, more and more game engines were developed to speed up the development of video games.
It should be mentioned that the definition of a game engine differs in the literature.
In this paper, game engines are seen as frameworks that accelerate the development process and provide at least support for 2D or 3D rendering.
It's also important to understand that not only is the definition very different, but so is the use case of a game engine.
ID Tech is a good example of how game engines can be developed specifically for a specific genre of video games.
There are also game engines like Unity, which could be called a general purpose game engine.
Therefore, a comparison between game engines is not easy.
Since game engines always serve a specific purpose, there is no game engine that stands above all others.
% TODO: needed? -> However, game engines can be compared to each other if the purpose is in focus.
This paper takes a closer look at the purpose of video game development in the indie game industry.
The Godot Engine in particular is brought into focus.
The motivation to examine exactly this game engine for this purpose comes from data sets, which will be explained in more detail in the course of this paper.

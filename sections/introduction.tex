\section{Introduction}
In the realm of video game development, the methodology for creating games has evolved significantly over time.
As early as the 1980s, several companies began using their own in-house engines.
For instance, in the case of Super Mario Bros (1985), Nintendo used a fast scrolling game engine that was originally developed for Excitebike (1984) \cite{history-digital-games}.
The concept of licensing game engines to other companies as a standalone product has been around since the mid-90's.
At that time, games like Doom (1993) were developed using the id Tech game engine \cite{id-tech-doom}.
This engine was specifically designed for 3D shooters, although this iteration was not yet a fully-fledged 3D game engine.
Over the years, an increasing number of game engines were developed to speed up the development of video games.
It should be mentioned that the definition of a game engine differs in the literature.
In this paper, game engines are seen as frameworks that accelerate the development process and provide at least support for 2D or 3D rendering.
It is also important to understand that not only is the definition very different, but so is the use case of a game engine.
A prime example is the id Tech engine, which was developed explicitly for a particular genre of video games.
There are also game engines like Unity, which could be called a general purpose game engine.
Therefore, a comparison between game engines is not easy.
Since game engines always serve a specific purpose, there is no game engine that stands above all others.
This paper takes a closer look at the purpose of video game development in the indie game industry, with a particular focus on the Godot Engine.
The decision to investigate this specific game engine is driven by the insights derived from a few datasets, which will be expounded upon later in this paper.

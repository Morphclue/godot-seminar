\section{Analysis}
In the last section it became clear that the Godot engine could attract the interest of larger companies.
This already indicates a certain relevance in the video game industry.
However, this does not yet point to the actual question of how relevant the Godot Engine is for indie game developers.
First, we have to define what exactly is meant by indie game development.
Just as the definition of a game engine is elusive, so is the definition of indie game development and indie games developed.
The definition used here is from O'Donnell, which states that indie games focus on a few goals and can be developed on a smaller scale in terms of manpower\cite{indie-definition}.
An indie game development team is not financially supported by a larger company.
This is in contrast to AAA titles, which are developed with high budgets and large teams.\\

For this paper, the digital sales platforms Steam and itch.io were selected for investigation.
On the one hand, this refers to a study by Toftedahl and Engström that provides comparative data from 2018\cite{game-engine-taxonomy}.
On the other hand, reference is made to a survey with 2700 video game developers.
The focus in this paper is on PC and not on console or mobile games.
This is because the survey ``State of the Game Industry'' from 2022 showed that 63\% are actively developing for PC and 58\% plan to do so in the next project as well\cite{gdc-2022}.
In 2019, a survey of video game developers with nearly 4000 participants\cite{gdc-2019} was conducted.
This survey showed that (besides publisher owned and own websites) most games are sold on Steam and itch.io.\\

Without exact details on Steam, it is difficult to track which game engine was used to develop a game.
In the paper by Toftedahl and Engström mentioned above, a script is used that matches the games with articles on Wikipedia and then assigns game engines.
Obviously, this procedure has problems and gaps in the data sets.
This is because not all games have information about the underlying game engine.
There is also another method, which recognizes the used game engines by filenames with the help of pattern matching.
The website SteamDB analyzes the game engines used for development based on this method\cite{steamdb-tech}.
There are problems with this method as well.
Gaps or false positives may exist due to unclear filenames.
In this paper, the datasets from SteamDB are used because matching filenames to game engines seems to be more accurate than matching the data to Wikipedia articles.\\

\begin{table}[h!]
    \centering
    \begin{tabular}{|c c c|}
        \hline
        Game Engine & 2018   & 2022    \\
        \hline\hline
        Unity       & 25.6\% & 61.22\% \\
        Unreal      & 13.2\% & 15.91\% \\
        Source      & 4.0\%  & 0.27\%  \\
        CryEngine   & 3.5\%  & 0\%     \\
        GameBryo    & 3.2\%  & 0\%     \\
        IW          & 2.9\%  & 0\%     \\
        Anvil       & 2.5\%  & 0\%     \\
        id Tech     & 1.7\%  & 0.18\%  \\
        Essence     & 1.1\%  & 0\%     \\
        Clausewitz  & 1.0\%  & 0.03\%  \\
        Other       & 48.4\% & 22.17\% \\
        \hline
    \end{tabular}
    \caption{Percentage of total games identified on Steam (data collected 2022-09-30)}
    \label{table:steam}
\end{table}

\autoref{table:steam} shows the percentages of game engines on Steam from 2022 compared to 2018 data.
In fact, with this comparison must be taken with a grain of salt because two different methods have been used.
It is also difficult to say whether the 2022 method can attribute more game engines or simply more games were created using a particular game engine.
This can be explained with the example of Unity.
Unity has 35.62\% more share of all identified games in the last four years.
This could be explained by the pattern matching method capturing more games.
However, it could also be the case that more games in general have been created using the Unity game engine. \\

% TODO: Graph Steam
% TODO: Explain other engines?

The Godot Engine is not listed in \autoref{table:steam} because the paper providing the comparison data does not list the Godot Engine.
At the time of the collected data, 1.15\% of all games created with the Godot Engine were identified.
Compared to the other game engines on SteamDB, the Godot engine ranks 7th out of 65 recognized game engines.
So there might be some relevance in the video game industry if the Godot Engine ranks so high.
% TODO: Graph X shows the games released per month.
If you look at the games released per month, you can clearly see that in 2020, the number of games released has increased significantly.\\

Before examining the reason in more detail, a similar comparison is first made with the data from itch.io.
Compared to Steam, the digital sales platform itch.io mainly features only indie games.
Since this paper focuses exactly on this type of games, this site provides an interesting data set.
Unlike Steam, itch.io has an official statistics website that provides information on game engines\cite{itchio-engines}.
However, this data is unfortunately also incomplete, because this data is self-reported.
So information could be missing or be incorrect here as well.

\begin{table}[h!]
    \centering
    \begin{tabular}{|c c c|}
        \hline
        Game Engine & 2018   & 2022                                                \\
        \hline\hline
        Unity       & 47.3\% & $\textcolor{ForestGreen}{\blacktriangleup 49.75\%}$ \\
        Construct   & 12.3\% & $\textcolor{ForestGreen}{\blacktriangleup 13.12\%}$ \\
        GameMaker   & 11.0\% & $\textcolor{red}{\blacktriangledown 7.32\%}$        \\
        Twine       & 6.2\%  & $\textcolor{red}{\blacktriangledown 5.35\%}$        \\
        RPG Maker   & 3.9\%  & $\textcolor{red}{\blacktriangledown 2.74\%}$        \\
        Bitsy       & 3.3\%  & $\textcolor{red}{\blacktriangledown 3.11\%}$        \\
        PICO-8      & 2.9\%  & $\textcolor{red}{\blacktriangledown 2.68\%}$        \\
        Unreal      & 2.8\%  & $\textcolor{ForestGreen}{\blacktriangleup 2.92\%}$  \\
        Godot       & 2.5\%  & $\textcolor{ForestGreen}{\blacktriangleup 5.55\%}$  \\
        Ren'Py      & 2.0\%  & $\textcolor{red}{\blacktriangledown 1.93\%}$        \\
        Other       & 5.9\%  & $\textcolor{red}{\blacktriangledown 5.55\%}$        \\
        \hline
    \end{tabular}
    \caption{Percentage of total games identified on itch.io (data collected 2022-09-30)}
    \label{table:itch}
\end{table}

Similar to Steam, itch.io data can be compared to the 2018 reference paper.
Since the method used to access the data is the same, a direct comparison can be made.
This comparison can be seen in \autoref{table:itch}.
Game engines that account for a higher percentage of the total games developed in the last four years are highlighted in green.
Game engines that performed worse are highlighted in red.
Comparing the data from \autoref{table:steam} and \autoref{table:itch}, it can immediately be seen that Unity seems to be a popular engine.
Unity accounts for over half of all games on Steam and almost exactly half on itch.io.
This is extremely high when compared to the second-placed game engines, which account for around 16\% on Steam and around 13\% on itch.io.\\

Furthermore, it can be seen that besides Unity, only three other game engines have gained popularity on itch.io.
These three game engines are Construct, Unreal and Godot Engine.
Construct is a game engine primarily aimed at non-programmers.
With the help of visual programming, it is intended to make it easier for novice programmers to get started.
The high percentage could be explained by the fact that many people are interested in developing their own games, but they lack the expertise to do so.
However, this is only an assumption that needs to be investigated further. \\

The Unreal Engine already shows up in \autoref{table:steam} as the second most used game engine on Steam.
However, on the itch.io platform, the Unreal Engine has only gained minimal popularity in the last four years.
This is particularly interesting because it shows that the Unreal Engine has a low relevance among indie game developers.
The Unreal Engine is often referred to as the game engine for AAA games\textcolor{red}{[?]}.
This could be the reason why indie game developers have less interest in it. \\

% TODO: add itch.io Graph

The Godot Engine has gained a lot of popularity in the last four years.
If you look at the underlying data, it becomes clear that the Godot Engine has more than double the percentage share than before.
% TODO: describe graph
% TODO: explain why unity excluded
% TODO: 2020 is important again
\\

The Godot engine can be classified as a general-purpose game engine, as it covers a wide range of game genres.
According to Toftedahl and Engström, this puts it in the same category as the Unreal Engine and Unity.
Construct, on the other hand, is a special purpose game engine.
In the introduction, it was already explained that game engines should be compared with each other in terms of a specific purpose.
For this reason, Construct should not be included in the comparison.

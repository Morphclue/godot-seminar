\section{Analysis}
In the last section it became clear that the Godot engine could attract the interest of larger companies.
This already indicates a certain relevance in the video game industry.
However, this does not yet point to the actual question of how relevant the Godot Engine is for indie game developers.
First, we have to define what exactly is meant by indie game development.
Just as the definition of a game engine is elusive, so is the definition of indie game development and indie games developed.
The definition used here is from O'Donnell, which states that indie games focus on a few goals and can be developed on a smaller scale in terms of manpower\cite{indie-definition}.
An indie game development team is not financially supported by a larger company.
This is in contrast to AAA titles, which are developed with high budgets and large teams.\\

For this paper, the digital sales platforms Steam and itch.io were selected for investigation.
On the one hand, this refers to a study by Toftedahl and Engström that provides comparative data from 2018\cite{game-engine-taxonomy}.
On the other hand, reference is made to a survey with 2700 video game developers.
The focus in this paper is on PC and not on console or mobile games.
This is because the survey ``State of the Game Industry'' from 2022 showed that 63\% are actively developing for PC and 58\% plan to do so in the next project as well\cite{gdc-2022}.
In 2019, a survey of video game developers with nearly 4000 participants\cite{gdc-2019} was conducted.
This survey showed that (besides publisher owned and own websites) most games are sold on Steam and itch.io.\\

Without exact details on Steam, it is difficult to track which game engine was used to develop a game.
In the paper by Toftedahl and Engström mentioned above, a script is used that matches the games with articles on Wikipedia and then assigns game engines.
Obviously, this procedure has problems and gaps in the data sets. 
This is because not all games have information about the underlying game engine.
There is also another method, which recognizes the used game engines by filenames with the help of pattern matching.
The website SteamDB analyzes the game engines used for development based on this method\cite{steamdb-tech}.
There are problems with this method as well.
Gaps or false positives may exist due to unclear filenames. 
In this paper, the datasets from SteamDB are used because matching filenames to game engines seems to be more accurate than matching the data to Wikipedia articles.


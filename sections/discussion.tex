\section{Discussion}
When examining various data sets, it becomes clear that the popularity of the Godot Engine has increased.
Starting with the digital sales platform Steam, it is noteworthy that the Godot Engine ranks 6th out of 65, which indicates its importance even at lower bounds, which arise due to data collection.
For a more meaningful discussion, Unity and Unreal Engine will also be included in this section.
According to \autoref{table:steam} Unity lost a few percent from 2022 to 2023 while Unreal Engine gained a few percent.
A possible reason for this could be that AAA titles are also published on Steam and Unreal Engine is known to be used for developing such games \cite{unreal-tripple-a-yager, unreal-tripple-a-india}.\\

On the other hand, this argument explains why growth on a platform like itch.io for Unreal Engine is rather slow compared to the Godot Engine.
While the Unreal Engine prominently appears as the second most used game engine on Steam, its presence on itch.io has seen only marginal growth of roughly 0.2\% over the past four to five years.
Furthermore, \autoref{table:itch} shows that during the same period of time, the Godot Engine has gained a share of roughly 5\%.
This is particularly interesting as it suggests that the Unreal Engine has limited relevance among indie developers, while the Godot Engine has become increasingly relevant in the indie game sector over time.
This observation is further supported by the fact that the Godot Engine ranks third on itch.io.\\

GameMaker is also included in this discussion, as this game engine has lost a large share of the sector.
Although Unity is the biggest player in the indie games sector, \autoref{table:itch} clearly shows that other game engines are gaining popularity for indie game development.
Another game engine under competitive pressure is GameMaker, which has lost roughly 4\% of its share over the years.
This observation can also be reproduced in the context of game jams, where Unity is gradually being displaced by other competitors, while GameMaker has been rapidly losing ground in recent years.
In contrast, the Godot Engine has seen a significant increase in popularity in game jams, where it has become the second most used game engine in the GMTK Game Jam.
It is important to consider that the overall popularity of a game engine could lead to its increased utilization in game jams.
Otherwise, it should also be noted that some games published in the GMTK Game Jam end up on itch.io and in this way also increase the number of games developed with the Godot Engine.
Further insight would require a precise analysis of this interaction. \\

Unity and GameMaker will persist as references in the Godot Engine Community Polls, as they are mentioned in relation to participants who have experience with these game engines.
Giving the declining shares of Unity and GameMaker, it is possible to assume that certain developers may have switched entirely to the Godot Engine.
There could be many reasons for this.
On one hand, the license of the Godot Engine offers great freedom and on the other hand, certain features could be favored.
However, in order to draw a direct comparison, a deeper analysis would also have to be carried out.
It should also be noted that not every game engine can be compared with every other, as mentioned at the beginning.
Each game engine has its own purpose and therefore a comparison is not easy.
For example one useful comparison would be between Unity and the Godot Engine, as both can be classified as general-purpose game engines that cover a wide range of game genres.\\

The results of this study make it clear that the Godot Engine definitely has relevance in the indie game industry.
In all three data sets analyzed (Steam, itch.io and GMTK Game Jams), the Godot Engine ranks better than most game engines.
In addition, it can also be said that various larger companies that have shown an interest in the Godot Engine through financial support indicating its relevance in the industry. 
The only question that remains is why this is the case.
Both \autoref{fig:godot-graph} and \autoref{fig:trend-graph} clearly show 2020 as a successful year for the Godot Engine.
If you combine this finding with the answers from the Godot Community Poll, you can see that people had already heard of the game engine before 2020, but only developed with it later.
It is also quite possible that events before 2020 were decisive for the rapidly growing popularity in the indie game industry.
However, this data reveals little about the reasons for this rise. Even after extensive research, it is difficult to pinpoint the exact reason for the growth.
This is probably due to the fact that many factors are responsible for this growth. \\

Another insight is that most games developed with the Godot Engine are mainly 2D games.
This is also in line with the statements from the Godot community survey.
On one hand, the Godot Engine might offer more robust features for 2D game development compared to its 3D capabilities.
On the other hand, the amount of work involved in developing 3D games is usually higher than that of 2D games, which can be a challenge for indie developers, causing them to favor 2D development. \\

The methods of analysis used here have raised many new questions.
These should continue to be investigated.
The market relevance in general or specifically for indie game studios could be further analyzed.
Possible suggestions include interviews with indie game studios that use the Godot Engine.
Additionally, in the case of Unity and GameMaker, it would be interesting to do further research to see if the Godot Engine is slowly pushing them out of the market.
One possible suggestion is to interview indie game studios that use the Godot Engine to gain further insights.
Furthermore, direct comparisons, where appropriate, would be useful to compare technical aspects between the Godot Engine and other game engines.
Methods such as a feature matrix, benchmarks or alternative methodologies would be useful for this. 
Although the reasons for Godot Engines growth in popularity are unclear, it can be concluded that the Godot engine is highly relevant in the indie game industry and will probably continue to grow over the next few years. 

\section{Methodology}
In the last section it became evident that the Godot Engine has managed to capture the attention of larger companies, suggesting a degree of relevance in the video game industry.
However, this does not directly address the central question of how relevant the Godot Engine is for indie game developers.
To answer this question effectively, we must first establish a clear definition of what indie game development is.
Just as the definition of a game engine is elusive, so is the definition of indie game development.
The definition used here is from O'Donnell, which states that indie games focus on a few goals and can be developed on a smaller scale in terms of manpower \cite{indie-definition}.
An indie game development team is not financially supported by a larger company.
This is in contrast to AAA titles, which are developed with high budgets and large teams.\\

To investigate the relevance of the Godot Engine for indie game developers, a comprehensive literature review was conducted.
It is important to note that this research was not limited to scholarly articles but also included gray literature.
This is because the Godot Engine is a relatively new game engine, and there is a lack of scholarly articles on the topic.
Nonetheless, the gray literature used in this paper was carefully selected to ensure that the information is reliable and trustworthy. \\ 

For this paper, the digital sales platforms Steam and itch.io were selected for investigation.
Firstly, it aligns with a study conducted by Toftedahl and Engström that provides comparative data from 2018 \cite{game-engine-taxonomy}.
Secondly, reference is made to a survey encompassing 2700 video game developers. % TODO: what survey + 2023?
Notably, the focus in this paper lies on PC and not on console or mobile games.
This is because the survey ``State of the Game Industry'' from 2022 showed that 65\% are actively developing for PC and 57\% plan to do so in the next project as well \cite{gdc-2023}.
In 2019, a survey involving nearly 4,000 video game developers highlighted that, aside from publisher-owned and individual websites, most games are primarily sold on Steam and itch.io \cite{gdc-2019}.\\

Identifying the game engine used in games available on Steam presents a significant challenge due to a lack of detailed information.
In the paper by Toftedahl and Engström mentioned above, a script was used that matches games with articles on Wikipedia and subsequently assigns game engines.
However, this procedure presents challenges and gaps in the data sets.
This is because not all games possess information about the underlying game engine. \\

Alternatively, another method involves recognizing the game engines used by analyzing filenames through pattern matching.
The website SteamDB employs this method to deduce the game engines utilized in development \cite{steamdb-tech}.
While this approach provides valuable insights, it is not without its limitations, including the potential for gaps or false positives due to ambiguous filenames.
In this paper, the datasets from SteamDB are used because matching filenames to game engines yield more accurate results than matching the data to Wikipedia articles. \\

Analyzing the data from itch.io proves to be more straightforward because the website provides a list of game engines used in development with the number of games for each engine. %TODO: cite
This structured presentation of data facilitates ease of access and analysis.
This approach aligns with the methodology employed by Toftedahl and Engström in their research.
However, it is important to acknowledge that this method captures a static snapshot of the data at a given point in time.
To address this limitation a script was developed that extracts data from itch.io's RSS feed in XML format, specifically recording the publication date of each game.
This additional temporal dimension allows for the creation of a time series, enabling a more in-depth analysis of used game engines over time.
